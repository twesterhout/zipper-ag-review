\documentclass[runningheads]{llncs}

\usepackage{amsmath}
\usepackage{calc}
\usepackage{csquotes}
\usepackage{color}
\usepackage{comment}
\usepackage{float}
\usepackage{graphicx}
\usepackage{url}
\usepackage{subcaption}
\usepackage{wrapfig}
\usepackage{pbox}

\captionsetup{compatibility=false}

%
%
\makeatletter
\@ifundefined{lhs2tex.lhs2tex.sty.read}%
  {\@namedef{lhs2tex.lhs2tex.sty.read}{}%
   \newcommand\SkipToFmtEnd{}%
   \newcommand\EndFmtInput{}%
   \long\def\SkipToFmtEnd#1\EndFmtInput{}%
  }\SkipToFmtEnd

\newcommand\ReadOnlyOnce[1]{\@ifundefined{#1}{\@namedef{#1}{}}\SkipToFmtEnd}
\usepackage{amstext}
\usepackage{amssymb}
\usepackage{stmaryrd}
\DeclareFontFamily{OT1}{cmtex}{}
\DeclareFontShape{OT1}{cmtex}{m}{n}
  {<5><6><7><8>cmtex8
   <9>cmtex9
   <10><10.95><12><14.4><17.28><20.74><24.88>cmtex10}{}
\DeclareFontShape{OT1}{cmtex}{m}{it}
  {<-> ssub * cmtt/m/it}{}
\newcommand{\texfamily}{\fontfamily{cmtex}\selectfont}
\DeclareFontShape{OT1}{cmtt}{bx}{n}
  {<5><6><7><8>cmtt8
   <9>cmbtt9
   <10><10.95><12><14.4><17.28><20.74><24.88>cmbtt10}{}
\DeclareFontShape{OT1}{cmtex}{bx}{n}
  {<-> ssub * cmtt/bx/n}{}
\newcommand{\tex}[1]{\text{\texfamily#1}}	% NEU

\newcommand{\Sp}{\hskip.33334em\relax}
\newcommand{\NB}{\textbf{NB}}
\newcommand{\Todo}[1]{$\langle$\textbf{To do:}~#1$\rangle$}

\EndFmtInput
\makeatother
%
%

%%format .*.       = "\mathbin{.\!\!*\!\!.}"
%%format .=.       = "\mathbin{.\!\!=\!\!.}"
%%format .<.       = "\mathbin{.\!\!<\!\!.}"
%%format sp = " "
%%format qq = "''\!"
%%format .$$ = ".\$_m"
%%format parent' = "parent_m"
%%format  <*>       = "\mathbin{\text{\small\ttfamily{<*>}}}"
%%format  <$>       = "\mathbin{\text{\small\ttfamily{<\$>}}}"
%%format t1 = "t_1"
%%format t2 = "t_2"
%%format constructor' = "constructor_m"
%%%%format Leaf = "{\sf Leaf}"
%%%%format Fork = "{\sf Fork}"
%%format Fork3 = "{\sf Fork3}"
%%%%%%format Root = "{\sf Root}"
%%format CRoot = "C_{\mathit{Root}}"
%%%%%%"{\sf C_{Root}}"
%%format CLeaf = "C_{\mathit{Leaf}}"
%%%%%"{\sf C_{Leaf}}"
%%format CFork = "C_{\mathit{Fork}}"
%%%%% "{\sf C_{Fork}}"
%%format lexeme_Leaf = "lexeme_{\sf Leaf}"
%%format MemoRoot = "{\sf Memo_{Root}}"
%%format MemoFork = "\mathit{Fork}_m"
%%%%%%"{\sf Fork}_m"
%%%%%% "{\sf Memo_{Fork}}"
%%format MemoLeaf = "\mathit{Leaf}_m"
%%%%%%"{\sf Leaf}_m"
%%%%%% "{\sf Memo_{Leaf}}"
%%format treeLookup = "lookup_{MT}"
%%%format MemoTable = "{Cache}"
%%format buildMTree = "\mathit{build}_m"
%%%%%%"build_{MT}"
%%format constructor_m = "\mathit{constructor}_m"
%%format tree_m = "\mathit{tree}_m"
%%format left_m = "\mathit{left}_m"
%%format right_m = "\mathit{right}_m"
%%format constructorM_m = "\mathit{constructorM}_m"
%%format treeM_m = "\mathit{treeM}_m"
%%format leftM_m = "\mathit{leftM}_m"
%%format rightM_m = "\mathit{rightM}_m"
%%format up_m = "\mathit{up}_m"
%%format down_m = "\mathit{down}_m"
%%format modify_m = "\mathit{modify}_m"
%%format mkAG_m = "mkAG_m"
%%format Cxt_m = "\mathit{Cxt}_m"
%%format Root_m = "Root_m"
%%format Top_m = "Top_m"
%%format L_m = "L_m"
%%format R_m = "R_m"
%%format MemoTree = "\mathit{Tree}_m"
%%format ZipperMemoTree = "\mathit{Zipper}_m"
%%format MemoAGTree = "\mathit{AGTree}_m"
%%format C_Memo_RootLet    = "C_{\mathit{Memo\_RootLet   }}"
%%format C_Memo_Let        = "C_{\mathit{Memo\_Let       }}"
%%format C_Memo_In         = "C_{\mathit{Memo\_In        }}"
%%format C_Memo_ConsAssign = "C_{\mathit{Memo\_ConsAssign}}"
%%format C_Memo_ConsLet    = "C_{\mathit{Memo\_ConsLet   }}"
%%format C_Memo_EmptyList  = "C_{\mathit{Memo\_EmptyList }}"
%%format C_Memo_Plus       = "C_{\mathit{Memo\_Plus      }}"
%%format C_Memo_Variable   = "C_{\mathit{Memo\_Variable  }}"
%%format C_Memo_Constant   = "C_{\mathit{Memo\_Constant  }}"
%%format lexemeConsAssign  = "lexeme_{\mathit{ConsAssign}}"
%%format lexemmeConsLet    = "lexeme_{\mathit{ConsLet}}"
%%format calculateMemo     = "calculate_{\mathit{Memo}}"
%%format errsAlgolMemo     = "errs_{\mathit{AlgolMemo}}"
%%format errorsMemo             = "errors_{\mathit{Memo}}"
%%format algol68                = "algol_{\mathit{68}}"
%%format ata_algol68            = "ata\_algol_{\mathit{68}}"
%%format LetSem.constructorMemo = "LetSem.constructor_{\mathit{Memo}}"
%%format lexemeString           = "lexeme_{\mathit{String}}"
%%format upGetVarValueMemo      = "up_{\mathit{GetVarValueMemo}}"
%%format lexemeInt              = "lexeme_{\mathit{Int}}"
%%format Child1              = "Child_{\mathit{1}}"
%%format Child2              = "Child_{\mathit{2}}"
%%format Child3              = "Child_{\mathit{3}}"
%%format Childi              = "Child_{\mathit{i}}"
%%format constructorMemo     = "constructor_{\mathit{Memo}}"
%%format at = "@"
%%format Algol68m               = "Algol68_{\mathit{m}}"

% Formatting
%%%%%%%%%%%%%%%%%%%%%%%%%%%%%%%%%%%%%%%%%%%%%%%%%%%%%%%%%%%%%%%%%%%%%%%%%%%%%%%%




\newcommand{\WithMath}[2]{{\parbox[][][b]{\widthof{#1}}{\centering$#2$}}}


% For the circular variant of repmin

% For the specification of AGs

% Tree positions


% A command for declaring todos
\newcommand{\TODO}[1]{{\color[rgb]{1,0,0}\textbf{TODO:}\textit{#1}}}

\newcommand{\ttsub}[2]{{\text{#1}_\text{#2}}}

\begin{document}

%%%%%%%%%%%%%%%%%%%%%%%%%%%%%%%%%%%%%%%%%%%%%%%%%%%%%%%%%%%%%%%%%%%%%%%%%%%%%%%%
\title{Modern type-safe embedding of attribute grammars}
\subtitle{An exercise with functional zippers in Haskell}


% TODO(twesterhout): This looks ugly... Someone, reformat it, please :)
\author{Jo{\~a}o Paulo Fernandes\inst{1}%
   \and Pedro Martins\inst{2}%
   \and Alberto Pardo\inst{3}%
   \and Jo{\~a}o Saraiva\inst{4}%
   \and Marcos Viera\inst{3}%
   \and Tom Westerhout\inst{5}%
}
\institute{%
  CISUC -- Universidade de Coimbra, Portugal \email{jpf@dei.uc.pt} \and
  University of California, Irvine, USA \email{pribeiro@uci.edu} \and
  Universidad de la  Rep\'{u}blica, Uruguay, \email{\{pardo,mviera\}@fing.edu.uy} \and
  Universidade do Minho, Portugal, \email{saraiva@di.uminho.pt} \and
  Radboud University, The Netherlands, \email{twesterhout@student.ru.nl}%
}

\date{}

\maketitle

\begin{abstract}
  % Introduction. In one sentence, what’s the topic?

  Attribute grammars are a powerful, declarative formalism to implement and
  reason about programs which, by design, are conveniently modular.

  % State the problem you tackle

  Although a full attribute grammar compiler can be tailored to specific needs,
  its implementation is highly non trivial, and its long term maintenance is a
  major endeavor.

  % Summarize (in one sentence) why nobody else has adequately answered the
  % research question yet.

  In fact, maintaining a traditional attribute grammar system is such a hard
  effort that most such systems that were proposed in the past are no longer
  active.

  % Explain, in one sentence, how you tackled the research question.

  Our approach to implement attribute grammars is to write them as first class
  citizens of a modern functional programming language.

  % In one sentence, how did you go about doing the research that follows from
  % your big idea.

  We improve a previous zipped-based attribute grammar embedding making it
  non-intrusive (i.e. no changes need to be made to the user-defined data types)
  and type-safe. On top of that, we achieve clearer syntax by using modern
  Haskell extensions.

  % As a single sentence, what’s the key impact of your research?

  We believe our embedding can be employed in practice to implement elegant,
  efficient and modular solutions to real life programming challenges.

\keywords{%
       Embedded Domain Specific Languages
  \and Zipper data structure
  \and Memoization
  \and Attribute Grammars
  \and Higher-Order Attribute Grammars
  \and Functional Programming%
}
\end{abstract}

%%%%%%%%%%%%%%%%%%%%%%%%%%%%%%%%%%%%%%%%%%%%%%%%%%%%%%%%%%%%%%%%%%%%%%%%%%%%%%%%
\section{Introduction}\label{sec:introduction}

  Attribute Grammars (AGs) are a declarative formalism that was proposed by
  Knuth~\cite{Knuth68} in the late 60s and allows the implementation and
  reasoning about programs in a modular and convenient way. A concrete AG
  relies on a context-free grammar to define the syntax of a language, and on
  attributes associated to the productions of the grammar to define the
  semantics of that language.
  %while adding \emph{attributes} to it so that it is also possible to define
  %its semantics.
  AGs have been used in practice to specify real programming languages, like
  for example Haskell \cite{DijkstraFS09}, as well as powerful pretty printing
  algorithms \cite{SPS99}, deforestation techniques \cite{joao07pepm} and
  powerful type systems \cite{MiddelkoopDS10}.

  When programming with AGs, modularity is achieved due to the possibility of
  defining and using different aspects of computations as separate attributes.
  Attributes are distinct computation units, typically quite simple and modular,
  that can be combined into elaborated solutions to complex programming
  problems. They can also be analyzed, debugged and maintained independently
  which eases program development and evolution.

  AGs have proven to be particularly useful to specify computations over
  trees: given one tree, several AG systems such
  as~\cite{syngen,uuag,lrc,silver} take specifications of which values, or
  attributes, need to be computed on the tree and perform these computations.
  The design and coding efforts put into the creation, improvement and
  maintenance of these AG systems, however, is tremendous, which often is an
  obstacle to achieving the success they deserve.

  An increasingly popular alternative approach to the use of AGs relies on
  embedding them as first class citizens of general purpose programming
  languages~\cite{Oege00,DBLP:conf/sblp/MartinsFS13,erlangAGs,kiama,doaitse09icfp,balestrieri}.
  This avoids the burden of implementing a totally new language and associated
  system by hosting it in state-of-the-art programming languages. Following this
  approach one then exploits the modern constructions and infrastructure that
  are already provided by those languages and focuses on the particularities of
  the domain specific language being developed.

  Functional zipper~\cite{huet1997zipper} is a powerful abstraction which greatly
  simplifies the implementation of traversal algorithms performing a lot of
  local updates. Functional zippers have successfully been applied to constuct
  an attribute grammar embedding in Haskell~\cite{DBLP:conf/sblp/MartinsFS13,MARTINS20162}. Despite its elegance,
  this solution had a major drawback which prevented its use in real-world
  applications: attributes were not cached, but rather repeatedly recomputed
  which severely hurt performance. Recently, this flaw has been
  eliminated~\cite{FERNANDES2018} and replaced with a different one: the approach became
  intrusive, i.e. to benefit from the embedding user-defined data structures
  have to be adjusted.

  In this paper we present an alternative mechanism to cache attributes based on
  a self-organising infinite grid. This graph is laid on top of the user-defined
  algebraic data type and mirrors its structure. The used-defined data type
  remains untouched. The embedding is then based on two coherent zippers (rather
  than one) traversing the data structures in parallel. On top of being
  non-intrusive our solution is completely type-safe. Modern Haskell extensions
  such as \texttt{ConstraintKinds} allow us to propagate constraints down in the
  ADT completely eliminating run-time type casts present in the previous
  versions. Another side benefit of using modern Haskell is a cleaner syntax
  with less code being generated with Template Haskell.


%%%%%%%%%%%%%%%%%%%%%%%%%%%%%%%%%%%%%%%%%%%%%%%%%%%%%%%%%%%%%%%%%%%%%%%%%%%%%%%%

\section{Functional Zippers}
  Zipper is a data structure commonly used in functional programming for
  traversal with fast local updates. The zipper data structure was originally
  conceived by Huet\cite{huet1997zipper} in the context of trees. We will,
  however, first consider a simpler problem: a bidirectional list traversal.

  \paragraph{Lists} Suppose that we would like to update a list at a specific
  position:
\begin{tabbing}\texfamily
~~modify~::~(a~\WithMath{->}{\rightarrow}~[a])~\WithMath{->}{\rightarrow}~{\itshape Int}~\WithMath{->}{\rightarrow}~[a]~\WithMath{->}{\rightarrow}~[a]\\
\texfamily ~~modify~f~i~xs~=~helper~[]~xs~0\\
\texfamily ~~~{\bfseries where}~helper~before~(x~:~after)~!j\\
\texfamily ~~~~~~~~~~~|~j~\WithMath{==}{\equiv}~i~~~~=~before~\WithMath{++}{+\!\!+}~f~x~\WithMath{++}{+\!\!+}~after\\
\texfamily ~~~~~~~~~~~|~\textit{otherwise}~=~helper~(before~\WithMath{++}{+\!\!+}~[x])~after~(j~+~1)\\
\texfamily ~~~~~~~~~helper~\char95 ~[]~\char95 ~=~\textit{error}~\char34 Index~out~of~bounds.\char34 
\end{tabbing}
  Here \text{\texfamily modify} takes an update action \text{\texfamily f}\footnote{\text{\texfamily f} returns a list rather
  than a single element to prevent curious readers from suggesting to use a
  boxed array instead of a list.}, an index \text{\texfamily i}, and a list \text{\texfamily xs} and returns a
  new list with the \text{\texfamily i}'th element replaced with the result of \text{\texfamily f}.%
  First, we ``unpack'' the list into $\text{\texfamily before\Sp =\Sp [}\ttsub{\text{\texfamily xs}}{\text{\texfamily 0}},\dots
  \ttsub{\text{\texfamily xs}}{\text{\texfamily i-1}}\text{\texfamily ]}$, $\text{\texfamily x\Sp =\Sp }\ttsub{\text{\texfamily xs}}{\text{\texfamily i}}$, and
  $\text{\texfamily after\Sp =\Sp [}\ttsub{\text{\texfamily xs}}{\text{\texfamily i+1}},\dots\text{\texfamily ]}$. Then replace \text{\texfamily x} by \text{\texfamily f(x)} and
  finally ``pack'' the result back into a single list. If we do a lot of
  updates, we end up unpacking and packing the list over and over again -- very
  time-consuming for long lists. We would thus like to benefit from fusion
  without explicitly working with the unpacked representation as it is
  bug-prone. A list zipper provides a way to achieve this.

  A zipper consists of a focus (alternatively called a hole) and surrounding
  context\footnote{`\text{\texfamily !}' in front of the type of \text{\texfamily \char95 cxt} is a
  strictness annotation enabled by the \text{\ttfamily BangPatterns} extension. Context can be
  seen as a path from the to the current focus and is always finite, i.e. there
  is no reason for it to be lazy.}:
\begin{tabbing}\texfamily
~~{\bfseries data}~{\itshape Zipper}~a~~=~{\itshape Zipper}~\char123 ~\char95 hole~::~a,~\char95 cxt~::~!({\itshape Context}~a)~\char125 \\
\texfamily ~~{\bfseries data}~{\itshape Context}~a~=~{\itshape Context}~[a]~[a]
\end{tabbing}
  where the \text{\texfamily {\itshape Context}} keeps track of elements to the left (\text{\texfamily before} in
  \text{\texfamily modify}) and to the right (\text{\texfamily after} in \text{\texfamily modify}) of the focus. We can now
  define movements:
\begin{tabbing}\texfamily
~~left~::~{\itshape Zipper}~a~\WithMath{->}{\rightarrow}~{\itshape Maybe}~({\itshape Zipper}~a)\\
\texfamily ~~left~({\itshape Zipper}~\char95 ~~~~({\itshape Context}~[]~\char95 ))~~~~~~~~=~{\itshape Nothing}\\
\texfamily ~~left~({\itshape Zipper}~hole~({\itshape Context}~(l~:~ls)~rs))~=~{\itshape Just}~\$\\
\texfamily ~~~~{\itshape Zipper}~l~\$~{\itshape Context}~ls~(hole~:~rs)\\
\texfamily \\
\texfamily ~~right~::~{\itshape Zipper}~a~\WithMath{->}{\rightarrow}~{\itshape Maybe}~({\itshape Zipper}~a)\\
\texfamily ~~right~({\itshape Zipper}~\char95 ~~~~({\itshape Context}~\char95 ~[]))~~~~~~~~=~{\itshape Nothing}\\
\texfamily ~~right~({\itshape Zipper}~hole~({\itshape Context}~ls~(r~:~rs)))~=~{\itshape Just}~\$\\
\texfamily ~~~~{\itshape Zipper}~r~\$~{\itshape Context}~(hole~:~ls)~rs
\end{tabbing}
  and functions for entering and leaving the zipper:
\begin{tabbing}\texfamily
~~enter~::~[a]~\WithMath{->}{\rightarrow}~{\itshape Maybe}~({\itshape Zipper}~a)\\
\texfamily ~~enter~[]~~~~~~~=~{\itshape Nothing}\\
\texfamily ~~enter~(x~:~xs)~=~{\itshape Just}~\$~{\itshape Zipper}~x~({\itshape Context}~[]~xs)\\
\texfamily \\
\texfamily ~~leave~::~{\itshape Zipper}~a~\WithMath{->}{\rightarrow}~[a]\\
\texfamily ~~leave~({\itshape Zipper}~hole~({\itshape Context}~ls~rs))~=~reverse~ls~\WithMath{++}{+\!\!+}~hole~:~rs
\end{tabbing}


  Finally, we define a local version of our \text{\texfamily modify} function (\TODO{Boy, is
  this function ugly...})
\begin{tabbing}\texfamily
~~modify~::~(a~\WithMath{->}{\rightarrow}~[a])~\WithMath{->}{\rightarrow}~{\itshape Zipper}~a~\WithMath{->}{\rightarrow}~{\itshape Maybe}~({\itshape Zipper}~a)\\
\texfamily ~~modify~f~({\itshape Zipper}~hole~({\itshape Context}~ls~rs))~=~{\bfseries case}~f~hole~{\bfseries of}\\
\texfamily ~~~~(x~:~xs)~\WithMath{->}{\rightarrow}~{\itshape Just}~\$~{\itshape Zipper}~x~({\itshape Context}~ls~(xs~\WithMath{++}{+\!\!+}~rs))\\
\texfamily ~~~~[]~~~~~~~\WithMath{->}{\rightarrow}~{\bfseries case}~rs~{\bfseries of}\\
\texfamily ~~~~~~(r~:~rs')~\WithMath{->}{\rightarrow}~{\itshape Just}~\$~{\itshape Zipper}~r~({\itshape Context}~ls~rs')\\
\texfamily ~~~~~~[]~~~~~~~~\WithMath{->}{\rightarrow}~{\bfseries case}~ls~{\bfseries of}\\
\texfamily ~~~~~~~~(l~:~ls')~\WithMath{->}{\rightarrow}~{\itshape Just}~\$~{\itshape Zipper}~l~({\itshape Context}~ls'~rs)\\
\texfamily ~~~~~~~~[]~~~~~~~~\WithMath{->}{\rightarrow}~{\itshape Nothing}
\end{tabbing}
  using which we can perform multiple updates efficiently and with minimal code
  bloat\footnote{%
  Operator \text{\ttfamily \char62{}\char61{}\char62{}} comes from \text{\texfamily {\itshape {\itshape Control}.Monad}} module in \text{\ttfamily base}
  and has the following signature:%
\begin{tabbing}\texfamily
~~(>=>)~::~{\itshape Monad}~m~=>~(a~\WithMath{->}{\rightarrow}~m~b)~\WithMath{->}{\rightarrow}~(b~\WithMath{->}{\rightarrow}~m~c)~\WithMath{->}{\rightarrow}~a~\WithMath{->}{\rightarrow}~m~c
\end{tabbing}
%
  }:
\begin{tabbing}\texfamily
~~modifyExample~::~{\itshape IO}~()\\
\texfamily ~~modifyExample~=~print~\$\\
\texfamily ~~~~enter~@{\itshape Int}~>=>~right\\
\texfamily ~~~~~~~~~~~~~~~~~>=>~right\\
\texfamily ~~~~~~~~~~~~~~~~~>=>~modify~(const~[])\\
\texfamily ~~~~~~~~~~~~~~~~~>=>~modify~(return~\WithMath{.}{\circ}~(+1))\\
\texfamily ~~~~~~~~~~~~~~~~~>=>~left\\
\texfamily ~~~~~~~~~~~~~~~~~>=>~modify~(return~\WithMath{.}{\circ}~negate)\\
\texfamily ~~~~~~~~~~~~~~~~~>=>~return~\WithMath{.}{\circ}~leave~\$~[1,~2,~3,~4,~5]
\end{tabbing}

  \paragraph{Trees} Consider now a binary tree data structure:
\begin{tabbing}\texfamily
~~{\bfseries data}~{\itshape Tree}~a~=~{\itshape Leaf}~!a~|~{\itshape Fork}~({\itshape Tree}~a)~({\itshape Tree}~a)
\end{tabbing}
  A binary tree zipper is slightly more insteresting than the list zipper,
  because we can move up and down the tree as well as left and right. The zipper
  again consists of a hole (a subtree we are focused on) and its surrounding
  context (a path from the hole to the root of the tree):
\begin{tabbing}\texfamily
~~{\bfseries data}~{\itshape Zipper}~a~~=~{\itshape Zipper}~\char123 ~\char95 hole~::~{\itshape Tree}~a,~\char95 cxt~::~!({\itshape Context}~a)~\char125 \\
\texfamily ~~{\bfseries data}~{\itshape Context}~a~=~{\itshape Top}\\
\texfamily ~~~~~~~~~~~~~~~~~|~{\itshape Left}~!({\itshape Context}~a)~({\itshape Tree}~a)\\
\texfamily ~~~~~~~~~~~~~~~~~|~{\itshape Right}~({\itshape Tree}~a)~!({\itshape Context}~a)
\end{tabbing}
  To move the zipper down, we ``unpack'' the current hole:
\begin{tabbing}\texfamily
~~down~::~{\itshape Zipper}~a~\WithMath{->}{\rightarrow}~{\itshape Maybe}~({\itshape Zipper}~a)\\
\texfamily ~~down~({\itshape Zipper}~({\itshape Leaf}~\char95 )~~~\char95 )~~~=~{\itshape Nothing}\\
\texfamily ~~down~({\itshape Zipper}~({\itshape Fork}~l~r)~cxt)~=~{\itshape Just}~\$~{\itshape Zipper}~r~({\itshape Right}~l~cxt)
\end{tabbing}
  \text{\texfamily {\itshape Context}} stores everything we need to reconstruct the hole, and \text{\texfamily up} does
  exactly that:
\begin{tabbing}\texfamily
~~up~::~{\itshape Zipper}~a~\WithMath{->}{\rightarrow}~{\itshape Maybe}~({\itshape Zipper}~a)\\
\texfamily ~~up~({\itshape Zipper}~\char95 ~{\itshape Top})~=~{\itshape Nothing}\\
\texfamily ~~up~({\itshape Zipper}~l~({\itshape Left}~cxt~r))~=~{\itshape Just}~\$~{\itshape Zipper}~({\itshape Fork}~l~r)~cxt\\
\texfamily ~~up~({\itshape Zipper}~r~({\itshape Right}~l~cxt))~=~{\itshape Just}~\$~{\itshape Zipper}~({\itshape Fork}~l~r)~cxt
\end{tabbing}
  Implementations of \text{\texfamily left}, \text{\texfamily right}, and \text{\texfamily enter} are very similar to the
  list zipper case and are left as an exercise for the reader. \text{\texfamily leave} differs
  slightly in that we now move all the way up rather than left:
\begin{tabbing}\texfamily
~~leave~::~{\itshape Zipper}~a~\WithMath{->}{\rightarrow}~{\itshape Tree}~a\\
\texfamily ~~leave~z~=~{\bfseries case}~up~z~{\bfseries of}\\
\texfamily ~~~~{\itshape Just}~z'~\WithMath{->}{\rightarrow}~leave~z'\\
\texfamily ~~~~{\itshape Nothing}~\WithMath{->}{\rightarrow}~\char95 hole~z
\end{tabbing}


  \paragraph{Generic Zipper} The list and binary tree zippers we have considered
  so far are homogeneous zippers: the type of the focus does not change upon zipper
  movement. Such a zipper can be a very useful abstraction. For example, a
  well-known window manager XMonad\cite{xmonad} uses a rose tree zipper to track
  the window under focus. For other tasks, however, one might need to traverse
  heterogeneous structures. A zipper that can accomodate such needs is usually
  called a \textit{generic zipper} as it relies only on the generic structure of
  Algebraic Data Types (ADTs). One can view an ADT as an Abstract Syntax Tree
  (AST) where each node is a Haskell constructor rather than a syntax construct.

  The generic zipper we will use is very similar to the one presented
  in\cite{adams2010syz}. The most common technique in Haskell for supporting
  heterogeneous types is Existential Quantification. However, not every type can
  act as a hole. To support moving down the tree, we need the hole to be
  \textit{dissectible}, i.e. we would like to be able to dissect the value into
  the constructor and its arguments. Even though \text{\texfamily {\itshape Data}.{\itshape Data}.gfoldl} allows us to
  achieve this, we define out own typeclass which additionally allows us to
  propagate down arbitrary constraints:
\begin{tabbing}\texfamily
~~{\bfseries class}~{\itshape Dissectible}~(c~::~{\itshape Type}~\WithMath{->}{\rightarrow}~{\itshape Constraint})~(a~::~{\itshape Type})~{\bfseries where}\\
\texfamily ~~~~dissect~::~a~\WithMath{->}{\rightarrow}~{\itshape Left}~c~a\\
\texfamily \\
\texfamily ~~{\bfseries data}~{\itshape Left}~c~expects~{\bfseries where}\\
\texfamily ~~~~{\itshape LOne}~~::~b~\WithMath{->}{\rightarrow}~{\itshape Left}~c~b\\
\texfamily ~~~~{\itshape LCons}~::~(c~b,~{\itshape Dissectible}~c~b)\\
\texfamily ~~~~~~~~~~=>~{\itshape Left}~c~(b~\WithMath{->}{\rightarrow}~expects)~\WithMath{->}{\rightarrow}~b~\WithMath{->}{\rightarrow}~{\itshape Left}~c~expects
\end{tabbing}
  For example, here is how we can make \text{\texfamily {\itshape Tree}} an instance of \text{\texfamily {\itshape Disssectible}}
\begin{tabbing}\texfamily
~~{\bfseries instance}~c~({\itshape Tree}~a)~=>~{\itshape Dissectible}~c~({\itshape Tree}~a)~{\bfseries where}\\
\texfamily ~~~~dissect~({\itshape Fork}~l~r)~=~{\itshape LOne}~{\itshape Fork}~`{\itshape LCons}`~l~`{\itshape LCons}`~r\\
\texfamily ~~~~dissect~x~=~{\itshape LOne}~x
\end{tabbing}
  We can unpack a \text{\texfamily {\itshape Fork}} and the zipper will thus be able to go down. \text{\texfamily {\itshape Leaf}}s,
  however, are left untouched and trying to go down from a \text{\texfamily {\itshape Leaf}} will return
  \text{\texfamily {\itshape Nothing}}.

  To allow the zipper to move left and right, we need a means to encode
  arguments to the right of the hole. Following Adams et al, we define a GADT
  representing constructor arguments to the right of the hole:
\begin{tabbing}\texfamily
~~{\bfseries data}~{\itshape Right}~c~provides~r~{\bfseries where}\\
\texfamily ~~~~{\itshape RNil}~~::~{\itshape Right}~c~r~r\\
\texfamily ~~~~{\itshape RCons}~::~(c~b,~{\itshape Dissectible}~c~b)\\
\texfamily ~~~~~~~~~~=>~b~\WithMath{->}{\rightarrow}~{\itshape Right}~c~provides~r~\WithMath{->}{\rightarrow}~{\itshape Right}~c~(b~\WithMath{->}{\rightarrow}~provides)~r
\end{tabbing}
  For example, for a tuple \text{\texfamily ({\itshape Int},\Sp {\itshape Int},\Sp {\itshape Int},\Sp {\itshape Int},\Sp {\itshape Int},\Sp {\itshape Int})}, we can have
\begin{tabbing}\texfamily
~~lefts~=~{\itshape LOne}~(,,,,,)~`{\itshape LCons}`~1~`{\itshape LCons}`~2~`{\itshape LCons}`~3\\
\texfamily ~~hole~=~4\\
\texfamily ~~rights~=~5~`{\itshape RCons}`~6~`{\itshape RCons}`~{\itshape RNil}
\end{tabbing}
  generalising this a little, we arrive at:
\begin{tabbing}\texfamily
~~{\bfseries data}~{\itshape LocalContext}~c~hole~rights~parent~=\\
\texfamily ~~~~{\itshape LocalContext}~!({\itshape Left}~c~(hole~\WithMath{->}{\rightarrow}~rights))~!({\itshape Right}~c~rights~parent)\\
\texfamily \\
\texfamily ~~{\bfseries data}~{\itshape Context}~::~({\itshape Type}~\WithMath{->}{\rightarrow}~{\itshape Constraint})~\WithMath{->}{\rightarrow}~{\itshape Type}~\WithMath{->}{\rightarrow}~{\itshape Type}~\WithMath{->}{\rightarrow}~{\itshape Type}~{\bfseries where}\\
\texfamily ~~~~{\itshape RootContext}~::~\char'24~c~root.~{\itshape Context}~c~root~root\\
\texfamily ~~~~(:>)~::~\char'24~c~parent~root~hole~rights.~(c~parent,~{\itshape Dissectible}~c~parent)\\
\texfamily ~~~~~~~~~=>~!({\itshape Context}~c~parent~root)\\
\texfamily ~~~~~~~~~\WithMath{->}{\rightarrow}~!({\itshape LocalContext}~c~hole~rights~parent)\\
\texfamily ~~~~~~~~~\WithMath{->}{\rightarrow}~{\itshape Context}~c~hole~root
\end{tabbing}
  And just like before the \text{\texfamily {\itshape Zipper}} is a product of the hole and context:
\begin{tabbing}\texfamily
~~{\bfseries data}~{\itshape Zipper}~(c~::~{\itshape Type}~\WithMath{->}{\rightarrow}~{\itshape Constraint})~(root~::~{\itshape Type})~=\\
\texfamily ~~~~\char'24~hole.~(c~hole,~{\itshape Dissectible}~c~hole)~=>\\
\texfamily ~~~~~~{\itshape Zipper}~\char123 ~\char95 zHole~::~!hole\\
\texfamily ~~~~~~~~~~~~~,~\char95 zCxt~~::~!({\itshape Context}~c~hole~root)\\
\texfamily ~~~~~~~~~~~~~\char125 
\end{tabbing}

  Implementation of movements is quite straightforward and is left out. Please,
  refer to (\TODO{github repo}) for complete code. We now consider a rather
  interesting application of generic zipper: embedding of attribute grammars.

%%%%%%%%%%%%%%%%%%%%%%%%%%%%%%%%%%%%%%%%%%%%%%%%%%%%%%%%%%%%%%%%%%%%%%%%%%%%%%%%


\section{Attribute Grammars}
  Attribute grammars (AGs) are an extension of context-free grammars that allow
  to specify context-sensitive syntax as well as the semantics. AGs achieve it
  by associating a set of attributes with each grammar symbol. These attributes
  are defined using evaluation rules assiciated with production rules of the
  context-free grammar.

  Attributes are then usually divided into two disjoint sets: synthesized
  attributes and the inherited attributes. Such distinction is required for the
  construction of a dependency graph. It is then used for specification of the
  evaluation order and detection of circularity. In the zipper-based embedding of
  attribute grammars we make no use of a dependency graph and thus do not divide
  attributes into classes.

  Let us consider the repmin problem as an example of a problem that requires
  multiple traversals:
  \begin{displayquote}
    Given a tree of integers, replace every integer with the
    minimum integer in the tree, in one pass.
  \end{displayquote}
  The classical solution is the following circular program:
\begin{tabbing}\texfamily
~~repmin~::~{\itshape Tree}~{\itshape Int}~\WithMath{->}{\rightarrow}~{\itshape Tree}~{\itshape Int}\\
\texfamily ~~repmin~t~=~t'\\
\texfamily ~~~~{\bfseries where}~(t',~m')~=~go~t~m'\\
\texfamily ~~~~~~~~~~go~({\itshape Leaf}~x~~~~)~m~=~({\itshape Leaf}~m,~x)\\
\texfamily ~~~~~~~~~~go~({\itshape Fork}~xs~ys)~m~=~({\itshape Fork}~xs'~ys',~min~\parbox[][3.5pt][t]{\widthof{mx}}{m$_{\text{x}}$}~\parbox[][3.5pt][t]{\widthof{my}}{m$_{\text{y}}$})\\
\texfamily ~~~~~~~~~~~~{\bfseries where}~(xs',~\parbox[][3.5pt][t]{\widthof{mx}}{m$_{\text{x}}$})~=~go~xs~m\\
\texfamily ~~~~~~~~~~~~~~~~~~(ys',~\parbox[][3.5pt][t]{\widthof{my}}{m$_{\text{y}}$})~=~go~ys~m
\end{tabbing}

  Although quite elegant, the code lacks modularity and is very difficult to
  reason about. Attribute Grammars provide a more modular approach. Viera et
  al\cite{viera2009agsfly} identified three steps for solving repmin: computing
  the minimal value, passing it down from the root to the leaves, and
  constructing the resulting tree. We can associate each step with an
  attribute\cite{swierstra1998designing}:
  \begin{itemize}
  \item A synthesized attribute \text{\texfamily localMin\Sp ::\Sp {\itshape Int}} represents the minimum value
        of a subtree. Computing the minimal value thus corresponds to evaluation
        of the \text{\texfamily localMin} attribute for the root tree.
  \item An inherited attribute \text{\texfamily globalMin\Sp ::\Sp {\itshape Int}} is used to pass down the
        minimal value.
  \item Finally, a synthesized attribute \text{\texfamily updated\Sp ::\Sp {\itshape Tree}\Sp {\itshape Int}} is the subtree
        with leaf values replaced by values of their \text{\texfamily globalMin} attributes. The
        solution is thus the value of \text{\texfamily updated} attribute for the root tree.
  \end{itemize}
  The obtained AG is presented in figure~\ref{fig:repmin-AG}. \TODO{Do we
  actually need to explain the algorithms here? It seems a little bit childish
  to explain how to compute the minimum of a binary tree... If we absolutely
  have to explain stuff, maybe just put it in the caption.}

\begin{figure}
\centering
\fbox{\parbox{\linewidth}{%
\begin{tabbing}\texfamily
~~SYN~{\itshape Tree}~{\itshape Int}~[localMin~:~{\itshape Int}]\\
\texfamily ~~SEM~{\itshape Tree}~{\itshape Int}~|~{\itshape Leaf}~lhs.localMin~=~@value\\
\texfamily ~~~~~~~~~~~~~~~|~{\itshape Fork}~lhs.localMin~=~min~@left.localMin\\
\texfamily ~~~~~~~~~~~~~~~~~~~~~~~~~~~~~~~~~~~~~~~~~@right.localMin\\
\texfamily \\
\texfamily ~~SYN~{\itshape Tree}~{\itshape Int}~[updated~:~{\itshape Tree}~{\itshape Int}]\\
\texfamily ~~SEM~{\itshape Tree}~{\itshape Int}~|~{\itshape Leaf}~lhs.updated~=~{\itshape Leaf}~@lhs.globalMin\\
\texfamily ~~~~~~~~~~~~~~~|~{\itshape Fork}~lhs.updated~=~{\itshape Fork}~@left.updated\\
\texfamily ~~~~~~~~~~~~~~~~~~~~~~~~~~~~~~~~~~~~~~~~~@right.updated\\
\texfamily \\
\texfamily \\
\texfamily ~~INH~{\itshape Tree}~{\itshape Int}~[~globalMin~:~{\itshape Int}~]\\
\texfamily ~~SEM~{\itshape Tree}~{\itshape Int}~|~{\itshape Fork}~left.globalMin~~=~@lhs.globalMin\\
\texfamily ~~~~~~~~~~~~~~~~~~~~~~right.globalMin~=~@lhs.globalMin\\
\texfamily \\
\texfamily ~~DATA~{\itshape Root}~|~{\itshape Root}~tree~:~{\itshape Tree}~{\itshape Int}\\
\texfamily ~~SEM~~{\itshape Root}~|~{\itshape Root}~tree.globalMin~=~@tree.localMin
\end{tabbing}
%
}}
\caption{%
  Attribute grammar for repmin. The syntax is closely mirrors the one used
  in\cite{swierstra1998designing}. \text{\texfamily SYN} and \text{\texfamily INH} introduce synthesized
  and inherited attributes respectively. \text{\texfamily SEM} is used for defining semantic
  rules. A new data type \text{\texfamily {\itshape Root}} is introduced as it is common in the AG setting
  to ``connect'' \text{\texfamily localMin} with \text{\texfamily globalMin}.
}
\label{fig:repmin-AG}
\end{figure}

  We now move on to embed this attribute grammar into Haskell. Semantic rules
  simply become functions, which, given a zipper, return values of the
  attributes. For example,
\begin{tabbing}\texfamily
~~localMin~::~{\itshape Zipper}~({\itshape WhereAmI}~{\itshape Position})~({\itshape Tree}~{\itshape Int})~\WithMath{->}{\rightarrow}~{\itshape Int}\\
\texfamily ~~localMin~z@({\itshape Zipper}~hole~\char95 )~=~{\bfseries case}~whereami~hole~{\bfseries of}\\
\texfamily ~~~~\parbox{\widthof{CLeaf}}{\textit{C}$_\text{\textit{Leaf}}$}~\WithMath{->}{\rightarrow}~{\bfseries let}~{\itshape Leaf}~x~=~hole~{\bfseries in}~x\\
\texfamily ~~~~\parbox{\widthof{CFork}}{\textit{C}$_\text{\textit{Fork}}$}~\WithMath{->}{\rightarrow}~{\bfseries let}~{\itshape Just}~l~=~child~0~z;~{\itshape Just}~r~=~child~1~z\\
\texfamily ~~~~~~~~~~~~~~{\bfseries in}~min~(localMin~l)~(localMin~r)
\end{tabbing}

  Apart from the type signature, the code is pretty straightforward and closely
  mirrors the AG we defined earlier. \text{\texfamily whereami} function allows us to ``look
  around'' and returns the position of the zipper. Thus the \text{\texfamily {\bfseries case}} corresponds
  to the pattern matches on the left of the vertical bars on
  figure~\ref{fig:repmin-AG}.

  Position of the zipper is encoded using the following GADT which can be
  generated automatically using Template Haskell:
\begin{tabbing}\texfamily
~~{\bfseries data}~{\itshape Position}~::~{\itshape Type}~\WithMath{->}{\rightarrow}~{\itshape Type}~{\bfseries where}\\
\texfamily ~~~~\parbox{\widthof{CLeaf}}{\textit{C}$_\text{\textit{Leaf}}$}~::~{\itshape Position}~({\itshape Tree}~{\itshape Int})\\
\texfamily ~~~~\parbox{\widthof{CFork}}{\textit{C}$_\text{\textit{Fork}}$}~::~{\itshape Position}~({\itshape Tree}~{\itshape Int})
\end{tabbing}
  Parametrization on the type of the hole allows the code like
  \text{\texfamily {\bfseries let}\Sp {\itshape Leaf}\Sp x\Sp =\Sp hole\Sp {\bfseries in}\Sp x} to typecheck, even though the generic zipper itself knows close
  to nothing about the type of the hole. It might seem trivial at first, because
  the binary tree zipper is in fact homogeneous. The ``position trick'' however
  extends also to heterogeneous zippers which we will encounter in more advanced
  examples.

\begin{tabbing}\texfamily
~~child~::~{\itshape Int}~\WithMath{->}{\rightarrow}~{\itshape Zipper}~cxt~root~\WithMath{->}{\rightarrow}~{\itshape Maybe}~({\itshape Zipper}~cxt~root)
\end{tabbing}
  \text{\texfamily child\Sp n} moves the zipper to the \text{\texfamily n}'th child, if there is one.


%%%%%%%%%%%%%%%%%%%%%%%%%%%%%%%%%%%%%%%%%%%%%%%%%%%%%%%%%%%%%%%%%%%%%%%%%%%%%%%%
\section{Related Work}\label{sec:related-work}

%%%%%%%%%%%%%%%%%%%%%%%%%%%%%%%%%%%%%%%%%%%%%%%%%%%%%%%%%%%%%%%%%%%%%%%%%%%%%%%%
\section{Conclusion}\label{sec:conclusion}

%%%%%%%%%%%%%%%%%%%%%%%%%%%%%%%%%%%%%%%%%%%%%%%%%%%%%%%%%%%%%%%%%%%%%%%%%%%%%%%%
\section*{Acknowledgements}

%%%%%%%%%%%%%%%%%%%%%%%%%%%%%%%%%%%%%%%%%%%%%%%%%%%%%%%%%%%%%%%%%%%%%%%%%%%%%%%%
% As per the LLNCS guidelines
\bibliographystyle{splncs04}
\bibliography{References}

\end{document}

